C++ 17 parameter parser like Perl\textquotesingle{}s Getopt\+::\+Long.

\subsection*{Usage}

\subsubsection*{Include}

First include the necessary header, wherever it may be hiding on your hard-\/drive.
\begin{DoxyCode}{0}
    \DoxyCodeLine{\#include <OptionsParser.hpp>}
\end{DoxyCode}
With including that one header you have all you {\itshape need} to work with Info\+Parse, more headers as described in the documentation are available, however messing with them is highly unadvised.

\subsubsection*{Instantiate}

In the scope in which you plan to use Info\+Parse in you need to instantiate an {\ttfamily Info\+Parse\+::\+Options\+Parse}. \mbox{[}Note\+: For the following parts of the documentation {\ttfamily namespace IP = Info\+Parse} is in place, as I\textquotesingle{}m too lazy to write that much.\mbox{]}
\begin{DoxyCode}{0}
\DoxyCodeLine{IP::OptionsParser parser;}
\end{DoxyCode}
{\ttfamily I\+P\+::\+Options\+Parser} is default-\/constructable and doesn\textquotesingle{}t offer any other constructor.

\subsubsection*{Optionize}

Having created the parser object you need to add the options you wish it to parse. The results are spit back into already allocated and constructed resources using a reference stored when adding the option. Option addition is achievable using the following method calls\+:
\begin{DoxyCode}{0}
\DoxyCodeLine{// Creates a "--string-long-opt" option with implicit "-s" short variant}
\DoxyCodeLine{// The string following the appearance of the option according to the}
\DoxyCodeLine{// local shell's partitioning will be stored in resultString. If}
\DoxyCodeLine{// option is not present, the empty string.}
\DoxyCodeLine{std::string resultString;}
\DoxyCodeLine{parser.addOption("string-long-opt", resultString); }
\DoxyCodeLine{}
\DoxyCodeLine{// Creates a "--long-bool-opt" option with "-b" short variant}
\DoxyCodeLine{// resultBool contains true if option present false otherwise.}
\DoxyCodeLine{bool resultBool;}
\DoxyCodeLine{parser.addOption("long-bool-opt", 'b', resultBool);}
\DoxyCodeLine{}
\DoxyCodeLine{// addOption accepts all types that overload operator>> for std::istream}
\DoxyCodeLine{// Otherwise a nice SFINAE error will greet you}
\DoxyCodeLine{MyType resultMyType;}
\DoxyCodeLine{parser.addOption("funky-option", 'f', resultMyType);}
\end{DoxyCode}


\subsubsection*{Parse}

After adding the options you may call parse any number of times you wish to be parsed according to those options. The parse method accepts either an std\+::string to be parsed which contains the options and parameters separated by whitespace as in a shell, or a char$\ast$$\ast$ and int parameters which work exactly as the ones passed into the main function.
\begin{DoxyCode}{0}
\DoxyCodeLine{int main(int argc, char** argv) \{}
\DoxyCodeLine{    //...}
\DoxyCodeLine{    std::string rem = parser.parse(argc, argv);}
\DoxyCodeLine{\}}
\end{DoxyCode}

\begin{DoxyCode}{0}
\DoxyCodeLine{std::string args;}
\DoxyCodeLine{std::string rem = parser.parse(args);}
\end{DoxyCode}
The {\ttfamily rem} string in both of these snippets contains the text remaining after removing the parsed text from the passed string (or c-\/array of c-\/strings).

\subsection*{Documentation}

W\+IP

\subsection*{License}

This project and repository is licensed under the B\+SD 3-\/Clause license. For more information check the provided {\ttfamily L\+I\+C\+E\+N\+SE} file.
